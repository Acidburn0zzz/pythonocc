%---------------------------Mean Aspect Frobenius-----------------------------
\section{Mean Aspect Frobenius}

For quadrilaterals, there is not a unique definition of the aspect Frobenius.
Instead, we use the aspect Frobenius
defined for triangles (see section~\S\ref{s:tri-aspect-Frobenius}).
Consider the four triangles formed by pairs of neighboring quadrilateral edges.
Given three counterclockwise, consecutively ordered quadrilateral vertices $i$, $j$, and $k$
denote the triangular aspect frobenius $F_{ijk}$.
To obtain a single value for the metric, we average the four unique triangular aspects
\[
  q = \frac{1}{4}\left(F_{301} + F_{012} + F_{123} + F_{230}\right).
\]

\quadmetrictable{mean aspect frobenius}%
{$1$}%                                      Dimension
{$[1,1.3]$}%                                Acceptable range
{$[1,DBL\_MAX]$}%                           Normal range
{$[1,DBL\_MAX]$}%                           Full range
{$1$}%                                      Unit square
{\cite{pebay:04}}%                          Citation
{v\_quad\_med\_aspect\_frobenius}%          Verdict function name

